\section{Computing distances}

In this example, we will explore how to compute distances on the GPU.
The \Rpkg{gputools} package already does this, so we will consider its
code.  Our aim is to contrast how this is done with \R{} bindings to
CUDA and how it is done with \C{} code.

The \Rfunc{gpuDist} function in the \Rpkg{gputools} package is similar
to \R's own \Rfunc{dist} function.  It takes a matrix of observations
and the name of a distance metric.  For ease of computation in the
\C{} code, the function transposes the matrix so that the elements of
each observation are contiguous in memory (i.e. in row order rather
than column order).  \Rfunc{gpuDist} calls a \C{} routine
\Cfunc{Rdistances}.  This is a simple wrapper that calls the routine
\Cfunc{distance}.  This routine copies the data from the host to the
device, allocating the space for the input and the output arrays.  It
then calls \Cfunc{distance_device} which launches the kernel.  Which
kernel is used depends on which distance metric is to be used.  So
\Cfunc{distance_device} contains the same code for each kernel.  It is
here that a kernel thread is launched for each pair of observations
and the GPU is actually involved in the computations.


We can invoke any of the kernels in the \Rpkg{gputools} packages
directly from \R and so remove the need for the \C{} routines
described above. This not only greatly simplifies the understanding
and development of the code, but also leads to more flexible, reusable
code.  We'll illustrate how to call the \Cfunc{euclidean_kernel_same}
kernel.  We start by extracting it from the \C{} code and compiling it
directly into PTX code (or cubin or fatbin format).  We then load this
into the \R{} session with \Rfunc{loadModule}.
This kernel expects a single matrix, say stored in \Rvar{AB}.
Mimicing the \C{} code in \Rpkg{gputools}, we pass the same
matrix as two different parameters. (This is related to a different
kernel we will  discuss below.)
Since we are passing it twice, we can transfer it just once to the GPU
memory and then pass a reference to that one instance of the data
in two places. This avoids making two copies of the data.
As a result, we copy the data to the device before calling
\Rfunc{.gpu} with
\begin{RCode}
ABref = copyToDevice(t(AB))
\end{RCode}
Note that we have transposed the matrix as the kernel
expects the elements of each observation to be contiguous.
This contrasts with \R's column-oriented representation of a matrix.

We can now launch the kernel. We pass the reference
to the matrix values (\Rvar{ABref}) and also
the stride/pitch between observations (the number of columns in 
the matrix) and the number of observations.
We pass these twice, eventhough the second set of parameters is not
used in the kernel.
We also pass space to insert the distances. This is an $n \times n$
matrix  (where $n$ is the number of rows in \Rvar{AB})
and we can pass a simple numeric vector with the correct number of
elements for this.  We'll convert it to a matrix after the kernel
completes.
We also need to specify the stride/pitch for this matrix which is
\Rexpr{nrow(AB)}, i.e. the number of columns in the distance matrix.
\begin{RCode}
d = .gpu(kernel,
         ABref, ncol(AB), nrow(AB),            # inputs & dimensions
         ABref, ncol(AB), nrow(AB),            # inputs again
         ncol(AB),                             # dimension of matrix
         dist = numeric(nrow(AB)^2), nrow(AB), # results
         2.0,                                  # ignored
         outputs = "dist",                # which arguments to copy back
         gridDim = c(nrow(AB), nrow(AB)), # square grid
         blockDim = 32L)                  # threads within block
\end{RCode}
The final three arguments in this call to \Rfunc{.gpu} are for
controlling \Rfunc{.gpu} rather than being passed to the kernel.
\Rarg{outputs} identifies which inputs(s) are to be copied back as part
of the result. These are the out variables of the kernel.
In our case, we only want the distances which we have explicitly named
\Rarg{dist}.
We run a separate block of threads for each pair of observations.
To do this, we specify a grid dimension of \Rexpr{c(nrow(AB),  nrow(AB))}.
The kernel uses $32$ threads to compute the squared differences of the
elements in the pair of observations and add them together.
This number is hard-coded in the kernel in order to use shared memory
between the cooperating threads.
Therefore, we use a block dimension of $32$.

Once the call to \Rfunc{.gpu} returns, the variable \Rvar{d} contains
the vector of all pairwise distances.  We can convert this to a matrix
and we have our result.  We should remove the variable \Rvar{ABref}
since we no longer need it. When we remove it, the garbage collector
can release the associated memory. This is on the GPU device and so we
do want to release it so that the memory is available for other tasks.

We mentioned that the second set of inputs are not actually used in
this kernel. As a result, we could specify \Rnull{} for the matrix
of values and any integer values.  \Rfunc{.gpu} transfers the \R{}
value to the kernel as the \C{} value \Cnull.
Ideally, the code should be changed to remove the parameters that are
not used. However, this happens when the code itself is complex to
develop and maintain and we don't want to modify it once it is working.
This is an argument for writing in a high-level
language such as \R{} and \Rpkg{RCUDA}.


\begin{comment}
Timings
See runDistTimes.R
For $224,970,001$ distances (14999^2),
R's \Rfunc{dist} takes 285.6 seconds (elapsed time).
On the Tesla K20, it takes 8.47 seconds.
This is is a speed-up factor of 33.7. And we see about 33 - 35 on different runs.
\end{comment}

% cudaMallocPitch
\subsection{Different approaches to allocating memory} 
One of the goals of writing the \Rpkg{RCUDA} package
is to allow \R{} programmers to explore and experiment
with the entire CUDA API. 
Writing \C{} code for all aspects of copying data between
the host and device and launching kernels is time-consuming and error-prone.
It makes one less likely to experiment with different approaches to
enhance performance. However, if we can perform experiments directly
in  \R{} code, we are more likely to take the time to understand what
idioms work best in different circumstances.

One example of using a different idiom arises in the \Rpkg{gputools}
package. Rather than just directly copying the matrix of values from
\R{} to the GPU's memory, the \C{} code uses \Cfunc{cudaMallocPitch}
and \Cfunc{cudaMemcpy2D} to transfer the data.  This has a potential
benefit of padding out the rows of the matrix so that they are aligned
in a way that makes the GPU more efficient. The GPU can then load
different blocks of memory in a more efficient manner.

We can change our implementation  to see if this does yield 
improved performance.
We can replace the call 
\begin{RCode}
ABref = copyToDevice(t(AB))
\end{RCode}
in our initial example with an explict
call to \Rfunc{cudaMallocPitch} and \Rfunc{cudaMemcpy2D}.
We do this with
%XXX make this work
\begin{RCode}
mem = cudaMallocPitch( ncol(AB) * 4L,  nrow(AB))
cudaMemcpy2D(mem$devPtr, mem$pitch, 
             convertToPtr(t(AB), 'float'), 
             ncol(AB)*4L, ncol(AB)*4L, nrow(AB),
             RCUDA:::cudaMemcpyHostToDevice)
\end{RCode}
Note that we explicitly convert the (transposed)
matrix to a \C-level array of \Ctype{float} elements.


We had to explicitly create this array as the memory we allocated
has no information about the type.
The \Rpkg{RCUDA} package provides a higher-level interface
for this, built on these SDK bindings.
We can use \Rfunc{mallocPitch} (without the cuda prefix)
and specify the type of element with
\begin{RCode}
mem = mallocPitch( ncol(AB), nrow(AB), "float")
\end{RCode}
With this function, we specify the number of elements and not the
number of bytes in the first argument.  Furthermore, the resulting
object returned by \Rfunc{mallocPitch} a class name \Rclass{PitchMemory2D}
and contains information about the location of the allocated memory,
its size and type of element.
This allows us to simply assign an \R{}  object to it and
have the assignment actually convert the \R{} object and copy it
to the GPU's memory.
So we can use
\begin{RCode}
mem[] = t(AB)
\end{RCode}
to copy the matrix to the GPU.
% Note that we access the memory and the pitch different as this is an
% S4 class. This is annoying, but still more convenient. 
% Dispatch for [<- isn't behaving 




% euclidean_kernel
\subsection{Distances between two matrices}
In some circumstances, we start with two matrices, say \Rvar{A} and
\Rvar{B}, and want to compute the distance between each observation in
\Rvar{A} and each observation in \Rvar{B}.  We could combine these two
matrices into one and compute all pairwise-distances between the
rows. However, this uses more memory by creating a new matrix.  It
also performs a large number of redundant computations that will be
ignored, namely the distances between each pair of rows in \Rvar{A}
and also in \Rvar{B} where we only want those for the observations
between \Rvar{A} and \Rvar{B}.
So we would like to do just the computations we want.

The \Rpkg{gputools} package contains a kernel that can handle two
different matrices as inputs.  This is called
\Cfunc{euclidean_kernel} (i.e. without the \Cfunc{_same} suffix.)
We can invoke this with 
\begin{RCode}
out = .gpu(mod$euclidean_kernel,
           t(A), ncol(A), nrow(A), 
           t(B), ncol(B), nrow(B), 
           ncol(A), 
           ans = numeric(nrow(A) * nrow(B)), nrow(A),
           outputs = "ans", 
           gridDim = c(nrow(A), nrow(B)),
           blockDim = 32L)
\end{RCode}
%$
The arguments are very similar to the previous kernel call,
but there are no ignored parameters.

Note that here we pass the two matrices directly in the call to
\Rfunc{.gpu}, rather than than copying them to the device ahead of the
call.  This is because we are not passing the same matrix as two
separate inputs.  Instead, we can leave it to the \Rfunc{.gpu}
function to copy the vectors to the device, pass the references
to each of them to the kernel, and then to release the memory 
after it is no longer used. This is more convenient.


\subsection{A different metric}
It is easy for us to use a different metric.  We can load a different
kernel and pass that to our function. 
For example, we can use the \Cfunc{canberra_kernel} in the
\Rpkg{gputools} code.
We can access it with 
\begin{RCode}
mod$canberra_kernel
\end{RCode}
%$
and then pass that to the call to \Rfunc{.gpu} function above without
changing any of the other arguments.  In this way, the kernel acts
like a function or a pointer to a routine in \C{} and allows us to
parameterize the same distance calculations.  This makes the code
simpler, as we would expect of a high-level implementation.  It also
makes it more flexible. If we want to use an entirely different
kernel, we can write just the code for that kernel, compile and load
it.  Our distance function does not need to know where the kernel came
from or have any a priori knowledge about it.


\subsection{Reusing the computed distances}
As we have seen before, it is sometimes useful to be able to leave the
results computed by one kernel on the device and have those be used as
inputs to a second kernel.  The \Rpkg{gputools} package uses this when
performing hierarhical clustering.  Starting with a collection of
observations (i.e. a data frame or matrix), we calculate the pair-wise
distances between all observations.  Then we pass these to the
\Rfunc{gpuHclust} function to compute the clusters.  If we just use
\Rfunc{gpuDist} and then \Rfunc{gpuHclust}, we wil have moved the
distances from the GPU device back to \R, and then copy them back to
the device.  \Rpkg{gputools} provides \Rfunc{gpuDistClust} to avoid
this unnecessary overhead .  This is implemented with a separate, but
very similar, routine to \Cfunc{distance}.  Again, this allocates and
copies the inputs on the GPU.  So there is a separate \R{} function
and a separate \C{} routine to implement this, adding to the
complexity.

We can also avoid the overhead in \Rpkg{RCUDA} using 
code of the form:
\begin{RCode}
distances = cudaMalloc(N["A"] * N["B"], elType = "numeric")
.gpu(mod$euclidean_kernel_same, .., distances, outputs = FALSE)
.gpu(mod$centroid_kernel, distances, ...) 
\end{RCode}
Here, we allocate the memory on the GPU for holding the computed
distances. We tell \Rfunc{.gpu} not to return it to \R.
At this point, the distances will be in this memory.
We then pass the reference to that memory on the device as the input
to the next kernel. 
This reduces the complexity and also allows the \R{} programmer
to combine the kernels in different ways than the original
developers planned.




